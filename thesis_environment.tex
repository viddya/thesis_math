%
% File thesis_environment.tex
%
% Description:
% This file contains environments for theorems, lemmas, definitions,
% proofs, corollaries, algorithms, examples, remarks, and assumptions.
% The numbering is set up so that the same counter is used for ALL
% these environments.  This seems to make it easier for readers to
% find referenced material.  The counter is reset for each chapter.
%
% EXAMPLE
%    Definition 3.1
%    Lemma 3.2
%    Example 3.3
%    Theorem 3.4
%    Theorem 3.5
%    Lemma 3.6 ....
%
%    Lemma 4.1
%    Corollary 4.2 ....
%
% The Definition environment is indented on both sides.  This
% is not required, but seemed to be a nice way to separate definitions
% from proven results.
%
\usepackage{graphicx}
\usepackage{floatrow}


\usepackage{listings}
\usepackage{color}

\definecolor{dkgreen}{rgb}{0,0.6,0}
\definecolor{gray}{rgb}{0.5,0.5,0.5}
\definecolor{mauve}{rgb}{0.58,0,0.82}

\lstset{frame=tb,
  language=C,
  %aboveskip=3mm,
  %belowskip=3mm,
  showstringspaces=false,
  columns=flexible,
  basicstyle={\small\ttfamily},
  numbers=none,
  numberstyle=\tiny\color{gray},
  keywordstyle=\color{blue},
  commentstyle=\color{dkgreen},
  stringstyle=\color{mauve},
  breaklines=true,
  breakatwhitespace=true,
  tabsize=3
}
%%%%%%%%%%%%%%%%%%%%%%%%%%%%%%%%%%%%
\newtheorem{thm}{Theorem}[chapter]
\newtheorem{lemma}[thm]{Lemma}
\newtheorem{cor}[thm]{Corollary}
\newtheorem{prop}[thm]{Proposition}
\newtheorem{defin}[thm]{Definition}
\newtheorem{obser}[thm]{Observation}
\newtheorem{remark}[thm]{Remark}
%%%%%%%%%%%%%%%%%%%%%%%%%%%%%%%%%%%%%%%%%%%%%%%%%%%%
\newenvironment{definition}{\begin{description}\begin{defin}}
{\end{defin}\end{description}\medskip}
\newenvironment{proof}{{\bf Proof:} }{\hfill\rule{2.1mm}{2.1mm}}
%%%%%%%%%%%%%%%%%%%%%%%%%%%%%%%%%%%%%%%%%%%%%%%%%%%%
\newfloatcommand{capbtabbox}{table}[][\FBwidth]
\floatsetup[table]{capposition=top}


\newcounter{example}
\newenvironment{example}{\refstepcounter{example}
         \medskip\par\noindent\textbf{Example \theexample.} } {\medskip}
