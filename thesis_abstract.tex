%
% File - thesis_abstract.tex
%
%
% Description:
% Write your abstract in this file.
%
%
%
The point-in-polygon (PIP) problem defines for a given set of points
in a plane whether those points are inside, outside or on the
boundary of a polygon. The PIP for a plane is a special case of point
range finding that has applications in numerous areas that deal with
processing geometrical data, such as computer graphics, computer
vision, geographical information systems (GIS), motion planning and
CAD. Point in range search can be performed by a brute force search
method, however solution becomes increasingly prohibitive when
scaling the input to a large number of data points and distinct
ranges (polygons). The core issue of the brute force approach is that
each data point must be compared to the boundary range requiring
both computation and memory access.  By using a spatial data structure
such as a quadtree, the number of data points computationally compared
within a specific polygon range can be reduced to be more efficient in
terms of performance on the CPU.  While Graphics Processing Unit (GPU) 
systems have the potential to advance the computational requirements
of a number of problems, their massive number of processing cores
execute efficiently only in certain rigid execution models such as
data-level parallel problems. The goal of this thesis is to demonstrate
that the GPU systems can still process irregular data structure such
as a quadtree and that it can be efficiently traversed on a GPU to find the points
inside a polygon. Compared with an optimized serial CPU implementation
based on the recursive Depth-First Search (DFS), the stack based
iterative Breadth-first search (BFS) on the GPU has a performance gain
of 3x to 500x.
\newpage

