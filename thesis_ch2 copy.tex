%
% File - ch2.tex
%
%
% Description:
% This file should contain the second real chapter or section of your
% thesis.
%
%
To understand how {\it AMG} performs on this problem, we first
examine the coarsening statistics in Table \ref{table:t3}. One
interesting phenomenon is that the relative density of the
operators (percentage of non-zeros) increases as the grids become
coarser. This increasing density can be seen in the last column,
where the average number of entries per row increases through
several coarse levels. Fortunately, the operator complexity is not
adversely affected: a short calculation shows that the operator
complexity is 2.97, while the grid complexity is 1.89. It is also
interesting to note that the first coarse-grid operator has more
actual nonzero coefficients than does the fine-grid operator. This
is relatively common for {\it AMG} on unstructured grids.

\medskip
\begin{table}
\begin{center}
\begin{tabular}{crrcc}\hline
       & Number  & Number of & Density  & Average entries \\
Level  & of Rows & nonzeros  & (\% full) & per row          \\
\hline
 0  &  4192  &  28832 & 0.002 & 6.9 \\
 1  &  2237  &  29617 & 0.006 & 13.2 \\
 2  &   867  &  14953 & 0.020 & 17.2 \\
 3  &   369  &   7345 & 0.054 & 19.9 \\
 4  &   152  &   3144 & 0.136 & 20.7 \\
 5  &    69  &   1129 & 0.237 & 16.4 \\
 6  &    30  &    322 & 0.358 & 10.7 \\
 7  &    20  &    156 & 0.390 &  7.8 \\
 8  &    18  &    125 & 0.383 &  6.9 \\
 9  &     3  &      9 & 1.000 &  3   \\\hline
\end{tabular}
\end{center}
\caption{\label{table:t3}{\em Statistics on coarsening for AMG
applied to frozen asphalt.}}
\end{table}

The convergence of the method is summarized in Table
\ref{table:t4}. Using the standard 2-norm, we see that after 11
V-cycles, the residual norm reached $10^{-7}$ and the process
attained an asymptotic convergence factor of 0.31 per $V$-cycle.
Clearly, {\it AMG} does not converge as rapidly for this problem
as for the previous model problem, where we saw a convergence
factor of about 0.1. Nevertheless, in light of the unstructured
grid and the strong, non-aligned jump discontinuities in the
coefficients, the convergence properties are quite acceptable.


\medskip
\begin{table}
\begin{center}
\begin{tabular}{ccc}\hline
          &                & Ratio of    \\
V-cycle &  $||{\bf r}||_2$ & $||{\bf r}||_2$\\\hline
   0 &  1.00e+00 &    --  \\
   1 &  4.84e-02 &   0.05 \\
   2 &  7.59e-03 &   0.16 \\
   3 &  1.72e-03 &   0.23 \\
   4 &  4.60e-04 &   0.27 \\
   5 &  1.32e-04 &   0.29 \\
   6 &  3.95e-05 &   0.30 \\
   7 &  1.19e-05 &   0.30 \\
   8 &  3.66e-06 &   0.31 \\
   9 &  1.12e-06 &   0.31 \\
  10 &  3.43e-07 &   0.31 \\
  11 &  1.05e-07 &   0.31 \\
  12 &  3.22e-08 &   0.31 \\\hline
\end{tabular}
\end{center}
\caption{\label{table:t4}{\em Convergence of AMG $V$-cycles
applied to popcorn in terms of the 2-norm (Euclidean norm) of the
residual.}}
\end{table}
